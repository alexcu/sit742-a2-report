\documentclass[12pt,titlepage]{article}

\usepackage{listings}

% Geometry
\usepackage{geometry}
\newgeometry{top=2.25cm,right=2cm,bottom=2.5cm,left=3cm}
% Graphics
\usepackage{graphicx}
\graphicspath{{images/}{../images/}}
% Times new roman package
\usepackage{mathptmx}
% Linespread of 1.5 as required by SIT
\linespread{1.5}
% URL
\usepackage{hyperref}
% Page rotation
\usepackage{pdflscape}
% CSV tables
\usepackage{csvsimple,booktabs,longtable}
% Define a command to read in CSVs (filename, caption, label)
\newcommand{\csvtable}[2]{
  \csvreader[
    longtable=rlcll,
    table head=
      \caption{#2}
      \label{tab:#1}\\
      \toprule
      \bfseries Sequence & \bfseries From & & \bfseries To & \bfseries Frequency \\
      \midrule
      \endfirsthead
      \caption*{Table \ref{tab:#1} (\textit{continued from Page~\pageref{tab:#1}}): #2}\\\toprule
      \bfseries Sequence & \bfseries From & & \bfseries To & \bfseries Frequency \\
      \midrule
      \endhead
      \bottomrule
      & & & & \textit{Continued on next page...}
      \endfoot
      \bottomrule
      \endlastfoot
  ]{csv/#1.csv}{1=\Sequence, 2=\From, 3=\To, 4=\Frequency}{\textbf{\Sequence} & \From & $\rightarrow$ & \To & \Frequency}
}
% Define a command to import the dot figure
\newcommand{\dotfigure}[2]{
  \newpage
  \begin{figure}[p]
    \centering
    \includegraphics[width=\textwidth]{figs/digraphs/#1}
    \caption[#2]{#2. Refer to Table~\ref{tab:#1} for frequency pattern interactions.}
    \label{fig:#1}
  \end{figure}
  \clearpage
  \newpage
}
% Code listings
\usepackage{listings}

\author{Alex Cummaudo \texttt{<ca@deakin.edu.au>}\\ Jake Renzella \texttt{<jake.renzella@deakin.edu.au>}\\
Deakin Software and Technology Innovation Laboratory\\School of Information Technology\\Deakin University, Australia}
\title{Hotel TULIP Web Server Data Analysis\\\normalsize{\bfseries Assignment 2 - SIT742 Modern Data Science}}

\begin{document}

\maketitle

\section*{Executive Summary}

This report summarises findings from a data exploration on the Hotel TULIP web server logs, recorded between the periods of August 2014 and August 2015. Each log contains one \textit{request}, or \textit{hit}, that lists fourteen attributes as described in the attached Data Dictionary spreadsheet. Publicly known client IP addresses were extracted from the MaxMind GeoIP2\footnote{See \url{http://dev.maxmind.com/geoip/geoip2/}.} dataset to analyse the location of requests (narrowed down to city). Additionally, user agent strings were parsed to analyse device and browser statistics using the Python \texttt{user-agents} library\footnote{See \url{https://pypi.python.org/pypi/user-agents}.}, thereby extrapolating demographics, usage trends, platform information, server performance, and security statistics from the raw logs provided in the dataset. Further details on the extraction of the data is provided in the source code attached in Appendix~\ref{apx:results}, and an interactive version of this file is published on \href{https://databricks-prod-cloudfront.cloud.databricks.com/public/4027ec902e239c93eaaa8714f173bcfc/7364378259770565/3552971541306612/8155742302574378/latest.html}{Databricks}.

\newpage

\tableofcontents\newpage
\listoffigures\newpage
%\listoftables\newpage

\newpage
\section{Key Findings}

A list of key findings in the analysis are as thus:

\begin{itemize}
  \item Foo
\end{itemize}

\newpage
\section{Introduction}

\section{Dataset}

\section{Method}

\subsection{Assumptions Made}

\subsection{Extraction Process}

\subsubsection{Internal/External IP Address Regular Expression}

To differentiate between private site visitors and external visitors, a regular expression was used to filter private/public IP address ranges. The regular expression is shown below:

\begin{lstlisting}
c_ip regexp 
	'(^127\.)|(^10\.) | 
	(^172\.1[6-9]\.) |
	(^172\.2[0-9]\.) |
	(^172\.3[0-1]\.) |
	(^192\.168\.)'
\end{lstlisting}

Using the not of the regular expression will select only public IP addresses.

\subsection{Data Mining}

\section{Results}

\subsection{Patterns Identified}

% INTERNAL IPs

% EXTERNAL IPs


\newpage
\appendix

\section{Additional Figures}

\clearpage


\section{Extrapolation Results}
\label{apx:results}

Attached on the following pages are the results from Databricks. You may also interact with this online on \href{https://databricks-prod-cloudfront.cloud.databricks.com/public/4027ec902e239c93eaaa8714f173bcfc/7364378259770565/3552971541306612/8155742302574378/latest.html}{Databricks}.


\end{document}
