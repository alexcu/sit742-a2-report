\subsection{Platform Categories}
\label{sec:results:platform}

The following section provides a breakdown of visitor sequence patterns by platform. The 4 platforms which are presented are PC (Personal Computers), Smartphones, Tablets and bots. The information presented can help TULIP Hotel understand which pages visitors from different platforms are viewing most. This can lead to actionable feedback such as which pages to optimise for mobile or tablet viewing.

\subsubsection{PC Visitors}
\label{sec:results:platform:pc}

Figure~\ref{fig:pc_requests} visualises PC user sessions. Table~\ref{tab:pc_requests}  (Appendix~\ref{app:additional_tables}) further extrapolates this data and identifies the frequency pattern values. The top patterns identified for PC visitors are:

\begin{enumerate}
  \item `Above and Beyond' \ra{} `Facilities' \ra{} `Rooms'
  \item `Facilities' \ra{} `About the Hotel' \ra{} `Rooms'
  \item `Above and Beyond' \ra{} `Rooms' \ra{} `Dining'
  \item `Facilities' \ra{} `Rooms' \ra{} `Dining'
  \item `Above and Beyond' \ra{} `Rooms' \ra{} `Offers'
  \item `Facilities' \ra{} `Rooms' \ra{} `Offers'
  \item `Above and Beyond' \ra{} `About' \ra{} `Rooms'
  \item `Above and Beyond' \ra{} `Facilities' \ra{} `Dining'
\end{enumerate}

Users tend to end their sessions on their PCs on `Offers', `Dining' and `Rooms'. The large variety of identified sequences suggests that PC users are more likely to generally browse the TULIP site. More pages are viewed and explored suggesting a more general approach to viewing the website. This highly differs compared to smartphone and even tablet users.

\paragraph{Proposed Suggestions} It is noticeable that PC users tend to have a wider exploration of the TULIP website than on other platforms. We propose that this wide array of navigation styles be presented to the user on the main homepage in a horizontal navigation panel, thereby emphasising these most frequently explored pages directly on desktop browsers.

\begin{landscape}
\pagestyle{empty}
\dotfigure{pc_requests}{Directional network graph visualising frequency patterns of requests made by PCs}{1.4}
\end{landscape}

\subsubsection{Smartphone Visitors}
\label{sec:results:platform:smartphone}

Figure~\ref{fig:smartphone_requests} visualises typical user sessions when on smartphones. Table~\ref{tab:smartphone_requests}  (Appendix~\ref{app:additional_tables}) further extrapolates this data and identifies the frequency pattern values. The top frequent page visit patterns are identified as on smartphones are:

\begin{enumerate}
  \item `Offers' \ra{} `Facilities' \ra{} `Dining'
  \item `Facilities' \ra{} `Guest Rooms' \ra{} `Dining'
  \item `Offers' \ra{} `Location \& Contacts' \ra{} `Dining'
  \item `Facilities' \ra{} `Location \& Contacts' \ra{} `Dining'
\end{enumerate}

Smartphone behaviours of user sessions are quite different from all other user patterns identified. This is possibly attributed to the considerably different viewing patterns made by smartphone users due to their `on-the-go' browsing.

Most smart phone users end at the `Dining' page. Interestingly, the smartphone category is the only category which includes the `Location \& Contacts' page---possibly indicating that users are looking for a phone number to book dining experiences directly on their smartphone. This is indicating that smartphone users are less interesting in booking or looking for accomodation details, and are more likely to use the site for dining information.

\paragraph{Proposed Suggestions} As users are most likely to find contact information on their smartphones, the smartphone UI could be refined to prioritise contact information, such as a telephone number. Additionally, users might be more inticed to visit the hotel if they are `on-the-go' and looking for a quick meal---prioritising the dining page on the homepage also may also be useful.

\dotfigure{smartphone_requests}{Directional network graph visualising frequency patterns of requests made on smartphones}{1}

\subsubsection{Tablet Visitors}
\label{sec:results:platform:tablet}

Figure~\ref{fig:tablet_requests} visualises the user sessions of patterns found on tablets. Table~\ref{tab:tablet_requests}  (Appendix~\ref{app:additional_tables}) further extrapolates this data and identifies the frequency pattern values. The top frequent page visit patterns are identified as on tablets are:

\begin{enumerate}
  \item `Facilities' \ra{} `Offers' \ra{} `Dining'
  \item `Facilities' \ra{} `Rooms' \ra{} `Offers'
  \item `Facilities' \ra{} `Rooms' \ra{} `Dining'
  \item `Above and Beyond' \ra{} `Offers' \ra{} `Dining'
  \item `Above and Beyond' \ra{} `Facilities' \ra{} `Rooms'
\end{enumerate}

Tablet sequence patterns are much more focused with less variety compared to PC and Smartphone browsing. This is possibly suggesting that users are behaving in a more focused matter with less variance. Most tablet users are ending at the `Dining' or `Offers' page. These patterns could also be attributed to the fact that there are less tablet users than the other categories.

\paragraph{Proposed Suggestions} As identified, tablet users are casual browsers. It would be useful to identify the facilities, rooms, offers and dining pages together and have them grouped on a tablet-friendly user interface. 


\dotfigure{tablet_requests}{Directional network graph visualising frequency patterns of requests made on tablets}{0.8}

\subsubsection{Bots Visitors}
\label{sec:results:platform:bots}

An analysis of bot sequence frequency can assist TULIP Hotel with information regarding improving Search Engine Optimisation (SEO), a vital technique for increasing user engagement and traffic. Of note could be that most bots start crawling in a sequence that follows from the page from `Our City' and end a sequence from the page `About the Hotel'. 

Figure~\ref{fig:bots_requests} visualises the crawling patterns of bots on the Hotel TULIP web server. Table~\ref{tab:bots_requests}  (Appendix~\ref{app:additional_tables}) further extrapolates this data and identifies the frequency pattern values. The top frequent crawling patterns are identified are:

\begin{enumerate}
  \item `Our City' \ra{} `Location and Contacts' \ra{} `About the hotel'
  \item `Home' \ra{} `Offers' \ra{} `About the Hotel'
  \item `Above and Beyond' \ra{} `Home' \ra{} `Offers'
  \item `Home' \ra{} `Rooms' \ra{} `About the Hotel'
  \item `Home' \ra{} `Rooms' \ra{} `Offers'
\end{enumerate}

\paragraph{Proposed Suggestions} Many bots typically crawl offers about the hotel, and this follows into information about the hotel (probably to cache information for search engine caching). It would be useful to promote crawling on the offers page by allowing bots to continuously crawl through this page, thereby allowing users to find offers using search engines.

\dotfigure{bots_requests}{Directional network graph visualising frequency patterns of requests made by bots}{1}